% !TeX root = report.tex
\usepackage[margin=2.5cm]{geometry}

\usepackage{amsmath,amssymb}  % For advanced math
\usepackage{iftex}
\ifPDFTeX
  \usepackage[T1]{fontenc}
  \usepackage[utf8]{inputenc}
  \usepackage{textcomp} % provide euro and other symbols
\else % if luatex or xetex
  \usepackage{unicode-math} % this also loads fontspec
  \defaultfontfeatures{Scale=MatchLowercase}
  \defaultfontfeatures[\rmfamily]{Ligatures=TeX,Scale=1}
\fi
\usepackage{lmodern} % Use the Latin Modern fonts
\usepackage{textcomp} % Provide euro and other symbols
  
% Use upquote if available, for straight quotes in verbatim environments
\IfFileExists{upquote.sty}{\usepackage{upquote}}{}
\IfFileExists{microtype.sty}{% use microtype if available
  \usepackage[]{microtype}
  \UseMicrotypeSet[protrusion]{basicmath} % disable protrusion for tt fonts
}{}

\makeatletter
\IfFileExists{parskip.sty}{\usepackage{parskip}}{% else
  \setlength{\parindent}{0pt}
  \setlength{\parskip}{6pt plus 2pt minus 1pt}}
\setlength{\emergencystretch}{3em} % prevent overfull lines
\providecommand{\tightlist}{%
  \setlength{\itemsep}{0pt}\setlength{\parskip}{0pt}}

\usepackage{graphicx} % For including graphics
\usepackage[pdfusetitle]{hyperref} % For hyperlinks in the document
\usepackage{lastpage} % To get the last page number
\usepackage{xcolor} % To use custom colors
\usepackage[bottom]{footmisc} % Put footnotes to bottom of the page

\usepackage{bookmark}
\IfFileExists{xurl.sty}{\usepackage{xurl}}{} % add URL line breaks if available
\urlstyle{same}

\usepackage{float} % Use float placement of figures
\floatplacement{figure}{H}

\usepackage{longtable,booktabs,array} % To enable tables spanning multiple pages
\usepackage{calc} % for calculating minipage widths
% Correct order of tables after \paragraph or \subparagraph
\usepackage{etoolbox}
\makeatletter
\patchcmd\longtable{\par}{\if@noskipsec\mbox{}\fi\par}{}{}
\makeatother
% Allow footnotes in longtable head/foot
\IfFileExists{footnotehyper.sty}{\usepackage{footnotehyper}}{\usepackage{footnote}}
\makesavenoteenv{longtable}

\usepackage{listings} % To enable code listings
\lstset{
    basicstyle = \fontsize{8}{10}\sffamily,
    %backgroundcolor=\color{light-gray},
    numbersep=5pt,
    numberstyle=\tiny\color{gray},
    xleftmargin=.25in,
    captionpos=b, % caption position
    breaklines=true
}
\definecolor{mygreen}{rgb}{0,0.6,0}
\definecolor{mygray}{rgb}{0.5,0.5,0.5}
\definecolor{mylightgraybg}{rgb}{0.95,0.95,0.95}
\definecolor{mymauve}{rgb}{0.58,0,0.82}
\lstdefinestyle{default}{
  backgroundcolor=\color{white},   % choose the background color; you must add \usepackage{color} or \usepackage{xcolor}; should come as last argument
  basicstyle=\footnotesize\ttfamily,        % the size of the fonts that are used for the code
  breakatwhitespace=false,         % sets if automatic breaks should only happen at whitespace
  breaklines=true,                 % sets automatic line breaking
  captionpos=b,                    % sets the caption-position to bottom
  commentstyle=\color{mygreen},    % comment style
  %deletekeywords={...},            % if you want to delete keywords from the given language
  escapeinside={\%*}{*)},          % if you want to add LaTeX within your code
  extendedchars=true,              % lets you use non-ASCII characters; for 8-bits encodings only, does not work with UTF-8
  firstnumber=1,                   % start line enumeration with line 1000
  frame=none,                      % adds a frame around the code (none, single)
  keepspaces=true,                 % keeps spaces in text, useful for keeping indentation of code (possibly needs columns=flexible)
  keywordstyle=\color{blue},    % keyword style
  keywords=[2]{True,False},
  keywordstyle={[2]\color{mygreen}},
  language=Python,                 % the language of the code
  morekeywords={*,...},            % if you want to add more keywords to the set
  numbers=left,                    % where to put the line-numbers; possible values are (none, left, right)
  numbersep=5pt,                   % how far the line-numbers are from the code
  numberstyle=\tiny\color{mygray}, % the style that is used for the line-numbers
  columns=flexible,
  rulecolor=\color{black},         % if not set, the frame-color may be changed on line-breaks within not-black text (e.g. comments (green here))
  showspaces=false,                % show spaces everywhere adding particular underscores; it overrides 'showstringspaces'
  showstringspaces=false,          % underline spaces within strings only
  showtabs=false,                  % show tabs within strings adding particular underscores
  stepnumber=1,                    % the step between two line-numbers. If it's 1, each line will be numbered
  stringstyle=\color{mymauve},     % string literal style
  tabsize=2,                       % sets default tabsize to 2 spaces
  title=\lstname                   % show the filename of files included with \lstinputlisting; also try caption instead of title
}
\lstset{style=default}

% Use pgfplots
\usepackage{pgfplots}
\pgfplotsset{compat=newest}
\usepgfplotslibrary{groupplots}
\usepgfplotslibrary{dateplot}

% Use circuitikz
\usepackage{circuitikz}

\usepackage[style=ieee,bibencoding=utf8,sorting=ydnt]{biblatex} % For bibliography
\addbibresource{references.bib} %Import the bibliography file
% When using nocite, disable warnings from biblatex
%\silence{package}{biblatex}

% Header and footer settings
\usepackage{fancyhdr} % For header and footer
\pagestyle{fancy}
\usepackage{extramarks}
\setlength{\headheight}{24.7pt}
\fancyhf{} % Clear default header and footer
%\fancyhead[L]{\leftmark}
\fancyhead[L]{\large\scshape\thetitle} % Left header
\fancyhead[R]{\includegraphics[width=2.8cm]{images/fh-kaernten-logo.png}} % Right header with logo
\fancyfoot[R]{\textit{\thepage/\pageref{LastPage}}} % Right footer
\fancyfoot[C]{\textit{Confidential}} % Centered footer
\renewcommand{\headrulewidth}{0.2pt}
\renewcommand{\footrulewidth}{0.2pt}

%\hypersetup{hidelinks}

\makeatletter
\newcommand{\makecustomtitle}{%
  {\Large{\textbf{\@title}}}
  \let\thetitle\@title  % Save the title for header
}
\makeatother

%\setcounter{secnumdepth}{-\maxdimen} % remove section numbering
