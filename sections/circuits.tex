\section{Circuits}\label{sec:circuits}

\begin{figure}
\centering
\begin{circuitikz}[scale=0.8, transform shape]
\draw (0,0) node (start) {}
to[sV=$V_i$] ++(0,2+\ctikzvalof{tubes/height})
to[C=$C_i$] ++(2,0) coordinate(Rg)
to[R=$R_g$] (Rg |- start)
(Rg) to[short,*-] ++(1,0)
node[triode,anchor=control] (Tri) {} ++(2,0)
(Tri.cathode) to[R=$R_c$,-*] (Tri.cathode |- start)
(Tri.anode) to [R=$R_a$] ++(0,2)
to [short] ++(3.5,0) node(Vatop) {}
to [V<=$V_a$] (Vatop |- start)
to [short] (start)
(Tri.anode) ++(0,0.2) to[C=$C_o$,*-o] ++(2,0)
(Tri.cathode) ++(0,-0.2) to[short,*-] ++(1.5,0) node(Cctop) {}
to[C=$C_c$,-*] (start -| Cctop);
\draw[red,thin,dashed] (Tri.north west) rectangle (Tri.south east);
\draw (Tri.east) node[right] {12AX7};
\end{circuitikz}
\caption{Triode amplifier}\label{fig:triode_amplifier}
\end{figure}

In Figure~\ref{fig:triode_amplifier} shows a circuit diagram of a triode tube amplifier.
