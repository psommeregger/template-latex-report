\section{Circuits}\label{sec:circuits}

\begin{figure}
\centering
\begin{circuitikz}[scale=0.8, transform shape]
\draw (0,0) node (start) {}
to[sV=$V_i$] ++(0,2+\ctikzvalof{tubes/height})
to[C=$C_i$] ++(2,0) coordinate(Rg)
to[R=$R_g$] (Rg |- start)
(Rg) to[short,*-] ++(1,0)
node[triode,anchor=control] (Tri) {} ++(2,0)
(Tri.cathode) to[R=$R_c$,-*] (Tri.cathode |- start)
(Tri.anode) to [R=$R_a$] ++(0,2)
to [short] ++(3.5,0) node(Vatop) {}
to [V<=$V_a$] (Vatop |- start)
to [short] (start)
(Tri.anode) ++(0,0.2) to[C=$C_o$,*-o] ++(2,0)
(Tri.cathode) ++(0,-0.2) to[short,*-] ++(1.5,0) node(Cctop) {}
to[C=$C_c$,-*] (start -| Cctop);
\draw[red,thin,dashed] (Tri.north west) rectangle (Tri.south east);
\draw (Tri.east) node[right] {12AX7};
\end{circuitikz}
\caption{Triode amplifier}\label{fig:triode_amplifier}
\end{figure}

In Figure~\ref{fig:triode_amplifier} shows a circuit diagram of a triode tube amplifier.
\subsection{Non-Inverting Amplifier}
\label{ssec:non-inverting-amplifier}

\begin{figure}
    \centering
    \begin{circuitikz}[american]
    \draw (0,0) node[above]{$V_{in}$} to[short, o-] ++(1,0)
    node[op amp, noinv input up, anchor=+](OA){\texttt{OA1}}
    (OA.-) -- ++(0,-1) coordinate(FB)
    to[R=$R_1$] ++(0,-2) node[ground]{}
    (FB) to[R=$R_2$, *-] (FB -| OA.out) -- (OA.out)
    to [short, *-o] ++(1,0) node[above]{$V_{out}$};
\end{circuitikz}

    \caption{A non-inverting op amp is an operational amplifier circuit with an output voltage that is in phase with the input voltage.}
    \label{fig:non-inverting-amplifier}
\end{figure}

The output voltage can then be given as in equation~\ref{eq:non-inverting-amplifier-output}.

\begin{equation}
V_{out} = V_{in}+(\frac{V_{in}}{R_1})R_2
\label{eq:non-inverting-amplifier-output}
\end{equation}

The gain is given by equation~\ref{eq:non-inverting-amplifier-gain}.
\begin{equation}
\frac{V_{out}}{V_{in}} = 1+(\frac{R_2}{R_1})
\label{eq:non-inverting-amplifier-gain}
\end{equation}


\subsection{Digital Inverter}
\label{ssec:inverter}

\begin{figure}
    \centering
    \begin{tikzpicture}
    \node[vcc] at (0, 4) {};
	\node[ground] at (0, 0) {};
    \node[nmos](M1) at (0, 1) {$M_1$};
	\node[nmos](M2) at (0, 3) {$M_2$};
    \draw (0, 0) -- (M1.E);
    \draw (0, 4) -- (M2.D);
    \draw (M2.E) to[short, -*] ++(0, -.25) coordinate(p1) -- (M1.D)
        (p1) to[short, -o] ++(1,0) node[above]{$V_{out}$};
    \draw (M2.G) to[short, -*] ++(0, -1) coordinate(p2) -- (M1.G)
        (p2) to[short, -o] ++(-.5,0) node[above]{$V_{in}$};
\end{tikzpicture}

    \caption{The ubiquitous digital inverter. The input voltage $V_{in}$ switches one
    of both transistors on, and the other is off.}\label{fig:inverter}
\end{figure}
